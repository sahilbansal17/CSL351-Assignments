\documentclass[12pt]{report}
\usepackage{graphicx, float, algorithm2e}
\usepackage[export]{adjustbox}

\title{\centering CSL351: Analysis and Design of Algorithms \\Assignment No. 2}
\author{\LARGE Sahil\\2016UCS0008}

% to use proper section numbering in the report type 
\renewcommand{\thesection}{\arabic{section}}

\begin{document}

\maketitle

%%%%%%%%%%%%%%%%%%%%%%%%%%%%%%%%%%%%%%%%%%%%%%%%
\section{Problem 1: Monotonically increasing functions:}



%%%%%%%%%%%%%%%%%%%%%%%%%%%%%%%%%%%%%%%%%%%%%%%%
\section{Problem 2: Identifying matching pair of socks:}
The smallest no. of gloves required to have atleast one matching pair in the best case is $2$ since we might end up picking the pair when we pick up the first two socks. 
\\ 
In the worst case, we need to pick atleast 10 socks to ensure that we pick atleast one matching pair. This is because there are 4 pair of red socks, 4 pairs of yellow and 1 pair of green. So, we might end up picking one sock from each pair during the first 9 times, i.e. we have 9 socks each from all 9 different pairs, so during the 10th time, we will obviously have 1 sock picked which forms a pair with the already picked ones. 

%%%%%%%%%%%%%%%%%%%%%%%%%%%%%%%%%%%%%%%%%%%%%%%%
\section{Problem 3: Selection sort analysis:}


%%%%%%%%%%%%%%%%%%%%%%%%%%%%%%%%%%%%%%%%%%%%%%%%
\section{Problem 4: Asymptotical analysis of maximum of two functions:}
It is given that $f(n)$ and $g(n)$ are asymptotically non-negative functions. We need to prove that 
\[max\left \{ f(n), g(n) \right \} = \Theta (f(n) + g(n))\]
\\
Using the basic definition of $\Theta$ notation: 
Let $f(n)$ and $g(n)$ be functions mapping positive integers to positive real numbers, we say that $f(n)$ is $\Theta(g(n))$ \textbf{iff} $f(n)$ is $O(g(n))$ and $f(n)$ is also $\Omega(g(n))$. 
Now, $f(n)$ is $O(g(n))$ \textbf{iff} 
\[\exists\ c > 0 \ \& \ n_{0} \geq 1 \ s.t. \ \forall n \geq n_{0}, \ f(n) \leq c \ g(n)\]
So, first we show that $max\left \{ f(n), g(n) \right \}$ is $O(f(n) + g(n))$. 
\\
Since, $f(n) \leq f(n) + g(n)$ and $g(n) \leq f(n) + g(n)$, thus, 
\[max\left \{ f(n), g(n) \right \} \leq f(n) + g(n) \ \forall n\]
This is because $max\left \{ f(n), g(n) \right \}$ has to be either of the two $f(n)$ or $g(n)$ depending on which one is larger. 
\\
Thus, $max\left \{ f(n), g(n) \right \}$ is $O(f(n) + g(n))$ where $c = 1$ and $n_{0} = 1$.
\\
\\
Now, we show that $max\left \{ f(n), g(n) \right \}$ is $\Omega(f(n) + g(n))$. 
\\
Since $f(n) \leq max\left \{ f(n), g(n) \right \}$ and $g(n) \leq max\left \{ f(n), g(n) \right \}$, thus, 
\[f(n) + g(n) \leq max\left \{ f(n), g(n) \right \} + max\left \{ f(n), g(n) \right \}\]
\[\Rightarrow f(n) + g(n) \leq 2*max\left \{ f(n), g(n) \right \}\]
\[\Rightarrow max\left \{ f(n), g(n) \right \} \geq \frac{1}{2} (f(n) + g(n)) \ \forall n\]
Thus, $max\left \{ f(n), g(n) \right \}$ is $\Omega(f(n) + g(n))$ where $c = 0.5$ and $n_{0} = 1$.
\\ \\
Having proved that $max\left \{ f(n), g(n) \right \}$ is $\Omega(f(n) + g(n))$ and also $O(f(n) + g(n))$, we can now conclude from the basic definition of $\Theta$ notation that, $max\left \{ f(n), g(n) \right \}$ is $\Theta(f(n) + g(n))$.

%%%%%%%%%%%%%%%%%%%%%%%%%%%%%%%%%%%%%%%%%%%%%%%%
\section{Problem 5: Brute force method for maximum subarray problem:}



%%%%%%%%%%%%%%%%%%%%%%%%%%%%%%%%%%%%%%%%%%%%%%%%

\end{document}
